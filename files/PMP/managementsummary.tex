\section{
	خلاصه مدیریتی
}

\subsection{
مسئله
}
با ظهور نگهانی پاندمی کرونا، بسیاری از کسب و کار‌ها و همچنین دانشگاه‌های سراسر ایران مجبور به تعطیلی شدند به همین علت بسیاری از امکانات دانشگاه‌ها نیز بی استفاده مانده‌اند. هم اکنون اما، زمزمه‌های بازگشایی دانشگاه‌ها به گوش می‌رسد بنابراین تمهیدات لازم به منظور استفاده مجدد و بهتر از امکانات باید مورد نظر قرار بگیرند. یکی از اینگونه امکانات بسیار موثر و مهم سالن‌های ورزشی دانشگاه است که با توجه به رویه کنونی به خوبی و متناسب با پتانسیل آن مورد استفاده دانشجویان قرار نمی‌گیرد. این پروژه قصد دارد با توسعه دادن یک سامانه برای رزرو سالن‌های ورزشی دانشگاه به صورت آنلاین به دانشجویان و اساتید و کارکنان دانشگاه برای دسترسی بهتر به امکانات ورزشی دانشگاه کمک کند.


\subsection{
برنامه و هدف
}
در این پروژه قصد داریم با توسعه یک برنامه تحت وب برای رزرو آنلاین سالن‌های ورزشی دانشگاه، دسترسی دانشجویان، اساتید و کارکنان دانشگاه را به امکانات ورزشی تسهیل کنیم. هدف از این پروژه بهبود دادن وضعیت سلامت جسمانی و روانی و نیز تقویت روحیه کار گروهی در افراد حاضر در دانشگاه است.


\subsection{
بازاریابی پروژه
}
این پروژه از سمت دانشگاه تهران سفارش داده شده‌است بنابراین مشتری اولیه خود را دارد. البته می‌توان پس از اتمام این پروژه با معرفی آن به دیگر دانشگاه‌های سراسر ایران، ماژول‌های هسته‌ای و لازم آن را به دیگر دانشگاه‌ها نیز فروخت. به منظور معرفی این سامانه به دیگر مشتری‌های احتمالی می‌توان پس از توسعه پروژه فعلی نظر کاربران را ثبت کرده و به همراه نمونه‌ سایت به آن‌ها ارائه نمود.