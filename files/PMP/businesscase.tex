\section{\lr{Business Case}}

\subsection*{
مقدمه و پیشینه
}

هدف سامانه رزرو سالن‌های ورزشی دانشگاه، ایجاد بستری برای امکان رزرو سالن‌های ورزشی دانشگاه از طرف دانشجویان، اساتید و کارکنان است به صورتی که این بستر قابل اتکا و قابل دسترسی آسان باشد. با توجه به بازگشایی تدریجی دانشگاه‌ها، نیاز به انجام تدارکات لازم به منظور آماده‌سازی فضای دانشگاه‌ها، از جمله سالن‌های ورزشی و ورزشگاه‌ها، بیش از پیش حس می‌شود. با پشت سر گذاشتن بیش از دو سال در خانه و بدون انجام فعالیت‌های ورزشی، پس از بازگشایی کامل، دانشجویان بیشتر از گذشته نیاز به استفاده از این امکانات را خواهند داشت.


\subsection*{
اهداف پروژه
}

از اهداف این پروژه می‌توان به ایجاد بستری پایدار و قابل اطمینان در شرایط متفاوت اینترنت و با قابلیت تعداد بالایی از کاربران همزمان اشاره کرد. همچنین بسیار حائز اهمیت است که این سامانه سرعت قابل قبول و دقت در زمان کوتاه وضعیت سالن‌های مختلف را برای کاربران خود به نمایش بگذارد. در این پروژه بحث مالی و سود چندان موضوعیت ندارد و هدف آن عمدتاً توسعه سامانه قابل اعتماد است که نیاز به آن بسیار حس می‌شود.


\subsection*{
وضعیت و مشکلات روند فعلی
}

در حال حاضر رزرو سالن‌های ورزشی به صورت حضوری صورت می‌پذیرد. متقاضیان هر سالن ورزشی باید برای رزرو آن‌ها به مسئولین هر یک از سالن‌ها به صورت جداگانه مراجعه کنند و درخواست خود را مطرح کنند. با توجه به آن که مسئولین همواره در دسترس نبوده و اطلاعات زمان‌های آزاد برای سالن‌های مختلف به نحو یکپارچه در اختیار دانشجویان، اساتید و یا کارکنان قرار داده نمی‌شود، وضعیت فعلی رضایت‌بخش نیست.\\
بنابراین با توجه به توضیحات گفته شده نیاز به یک سامانه یکپارچه و قابل اطمینان محسوس است. این سامانه برای آن که بتواند توسط تمام کاربران با دستگاه‌های مختلف قابل استفاده باشد، به صورت اپلیکیشن تحت وب خواهد بود و اعلانات را به کاربران از طریق ایمیل اطلاع رسانی می‌کند تا روشی تاثیرگذار برای رفع نیاز را به همراه داشته باشد.


\subsection*{
فرضیات و شروط مهم
} 

این سامانه باید در شرایط کاهش سرعت اینترنت به کار خود ادامه دهد و در صورت قطعی اینترنت، درخواست کاربر برای رزرو سالن را ذخیره کرده تا پس از اتصال مجدد آن را برای سرور ارسال کند. یکپارچگی داده‌های ذخیره شده از زمان‌های رزرو شده سالن‌ها توسط افراد، در پایگاه‌های داده بسیار مهم و حیاتی است. همچنین باید بتوان با استفاده از سرورهای کنونی دانشگاه این سامانه را راه‌اندازی کرد. سامانه نباید نیاز روزانه به تکنسین‌ها داشته باشد؛ زیرا هزینه‌ آن بسیار هنگفت خواهد شد و عملاً چنین چیزی برای دانشگاه قابل قبول نیست. با توجه به احتمال بالای حضوری بودن ترم آینده تحصیلی، این سامانه باید در مدت ۳ ماه راه اندازی شود تا برای ترم آینده قابل بهره برداری باشد.


\subsection*{
تحلیل گزینه‌ها و پیشنهادات
}

گزینه‌های موجود عبارتند از:
\begin{itemize}
	\item {
 رویه کنونی ادامه پیدا کرده و کاری انجام ندهیم.	
	}
	\item  {
 رویه کنونی ادامه پیدا کرده و علاوه‌ بر آن هر دانشکده اطلاعیه‌هایی در رابطه با روز‌هایی که سالن‌های دانشکده قابل رزرو هستند در اختیار دانشجویان، اساتید و کارکنان قرار دهد.	
	}
	\item {
سامانه جدیدی با امکانات هماهنگ سازی میان دانشکده‌ها و با قابلیت‌های بسیار بیشتر راه‌اندازی کنیم.
	}
\end{itemize}

با توجه به نیازمندی موجود به سامانه‌ای با امکانات و کارایی بیشتر و همچنین با قابلیت اطلاع رسانی همگانی و یکپارچه سازی بین دانشکده‌ای، گزینه سوم گزینه مناسب‌تری است. یکپارچه‌سازی امکان اینکه افراد از دانشکده‌های مختلف از سالن‌های دیگر دانشگاه‌ها نیز استفاده کنند را فراهم می‌کند، امکانی که در دیگر گزینه‌ها به جز گزینه سوم موجود نیست.


\subsection*{
الزامات اولیه پروژه
}

\begin{enumerate}
	\item{
کاربران (شامل دانشجویان، اساتید و کارکنان دانشگاه) بتوانند با استفاده از اطلاعات احراز هویت دانشگاهی خود به این سامانه وارد شوند.
	}
	
	\item{
	مسئولین سالن‌های مختلف بتوانند ساعات کاری سالن‌ها را در سامانه وارد کنند و در صورت تغییر این ساعات به دیگر کاربران اطلاع رسانی شود.
	}
	
	\item{
همه کاربران بتوانند تمامی سالن‌های موجود در دانشگاه، اعم از دانشکده‌های مختلف را روی نقشه مشاهده کنند. همچنین با کلیک کردن روی نام سالن‌ها اطلاعات و تصاویری برای کاربران نمایش داده شود.	
	}
	
	\item {
همه کاربران بتوانند با کلیک کردن روی سالن‌ها زمان‌های کاری سالن و ساعاتی که توسط دیگران رزرو شده‌اند و ساعات آزاد را مشاهده کنند.	
	}
	
	\item {
	پس از انتخاب ساعت مورد نظر توسط کاربر برای رزرو سالن، در صورت عدم تداخل و رعایت شرایط لازم (در بازه زمانی کاری سالن بودن زمان انتخابی)، سالن به نام کاربر برای مدت زمان محدود رزرو شود.
	}
	
	\item {
	کاربران امکان لغو رزرو خود را داشته باشند.
	}
	
	\item {
	در صورت لغو رزرو برای سالن خاص، به افرادی که در زمان‌های نزدیک همان سالن را رزرو کرده‌اند، اطلاع رسانی شود.	
	}
	
	\item {
	امکان اعمال محدودیت تعداد رزرو در هفته برای نقش‌های مختلف (دانشجو، استاد، کارکنان) در سیستم وجود داشته باشد.
	}
	
	\item {
امکان جستجو میان سالن‌ها برای تمام کاربران بر اساس نام سالن و دانشکده وجود داشته باشد.	
	}
	
	\item {
	امکان اعمال فیلتر‌های متفاوت برای جستجو میان سالن‌ها وجود داشته باشد.
	}
	
	\item {
	امکان اعلام اعلانات از طریق ایمیل به کاربران وجود داشته باشد.
	}
\end{enumerate}


\subsection*{
تخمین بودجه و هزینه‌ها
}

با توجه به زمان کمی که این پروژه باید در آن تحویل داده شود و نیز آن‌ که پروژه بستر آماده‌ای در حال حاضر ندارد و باید به طور کامل توسعه پیدا کند، در مجموع این پروژه نیاز به ۱۵ نفر عضو تمام وقت و ۲ نفر عضو پاره وقت دارد. هر عضو تمام وقت ماهانه ۱۹۲ ساعت کار کرده و هر عضو پاره وقت ماهانه ۱۰۰ ساعت کار می‌کند. در این میان مدیر پروژه به صورت ماهانه ۲۵ میلیون تومان دریافتی خواهد داشت و مدیران تیم فرانت اند و بک اند ۱۳ میلیون تومان، مدیر تیم فناوری اطلاعات ۱۷ میلیون تومان به صورت ماهانه دریافت خواهند کرد. دیگر افراد تمام وقت تیم به صورت میانگین ۸ میلیون تومان و افراد پاره وقت ۵ میلیون دریافت خواهند کرد. در نهایت تخمین هزینه این پروژه حدود ۶۵۰ میلیون تومان خواهد بود.


\subsection*{
تخمین زمانی
}

با توجه به نیازمندی دانشگاه به سامانه رزرو سالن‌های ورزشی و با توجه به شرایط تیم توسعه، توسعه این سامانه ۳ ماه به طول می‌انجامد و در بهمن ماه ۱۴۰۰ قابلیت بهره‌برداری خواهد داشت.


\subsection*{
ریسک‌های احتمالی
}

ممکن است از طرف جامعه هدف (افراد دانشگاهی) از این سامانه استقبال صورت نگیرد. همچنین با توجه به هزینه‌های لازم برای توسعه سامانه، ممکن است پرداخت این هزینه برای دانشگاه مقدور نباشد (با توجه به شرایط کرونایی کنونی و کاهش درآمد دانشگاه). ممکن است با توجه به کوتاهی زمان پروژه، نتیجه مطلوب حاصل نشود و تا زمان مشخص شده شرایط مطلوب برای استفاده از این سامانه به وجود نیاید. در نهایت ممکن است ترم آینده نیز به صورت آنلاین برگزار شود و به تبع آن از این سامانه استفاده نشود.


\subsection*{
نمود پروژه
}

نمود پروژه یک سایت رزرو کردن سالن‌های ورزشی دانشگاه است که امکان ورود به سیستم در آن وجود دارد و به سامانه احراز هویت دانشگاه متصل می‌شود. در این سامانه کاربران می‌توانند تمام سالن‌های موجود اطلاعات مرتبط به هر یک را مشاهده کنند. مسئولان سالن‌های مختلف می‌توانند زمان‌های کاری این سالن‌ها را تعیین کنند. کاربران سامانه می‌توانند سالن‌های رزرو نشده را در زمان‌های کاری رزرو کنند.
